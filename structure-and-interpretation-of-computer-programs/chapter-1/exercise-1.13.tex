\documentclass{article}

\usepackage{amsmath}

\newcommand{\Fib}[1][]{\ensuremath{\text{Fib}(#1)}}
\newcommand{\tif}{\ensuremath{\text{if} \;}}
\newcommand{\s}{\sqrt{5}}

\begin{document}

\[
	\Fib[n] = \begin{cases}
		0                         & \tif n = 0 \\
		1                         & \tif n = 1 \\
		\Fib[n - 1] + \Fib[n - 2] & \text{otherwise} 
	\end{cases}
\]

By induction prove that:
\[
	\Fib[n] = \frac{(\varphi^n - \psi^n)}{\s}
\]

\section{Base case}

For $n = 0$
\[
	\begin{aligned}
   	\Fib[0] = 0 \\
		\frac{\varphi^0 - \psi^0}{\s} = 0
	\end{aligned}
\]

For $n = 1$

\[
	\Fib[1] = 1 
\]
\[
	\begin{aligned}
		\frac{\varphi^1 - \psi^1}{\s}  &= \frac{(1 + \s) - (1 - \s)}{2\s}  \\
		& = \frac{2 \s}{2 \s} \\
		& = 1 
	\end{aligned}
\]

For $n = 2$

\[
	\Fib[2] = 1 
\]
\[
	\begin{aligned}
		\frac{\varphi^2 - \psi^2}{\s}  &= \frac{(1 + \s)^2 - (1 - \s)^2}{2^2\s}  \\
		& = \frac{1 + 2\s + \s^2 - (1 - 2\s + \s^2)}{4 \s} \\
		& = \frac{2\s + 2\s}{4 \s} \\
		& = \frac{4\s}{4 \s} \\
		& = 1 
	\end{aligned}
\]

True. 

\section{Inductive step:}

Assume 
\[
	\Fib[n] = \frac{(\varphi^n - \psi^n)}{\s}
\]
\[
	\Fib[n - 1] = \frac{\varphi^{n - 1} - \psi^{n - 1}}{\s}
\]

Is it true that:
\[
	\Fib[n - 1] = \frac{\varphi^{n + 1} - \psi^{n + 1}}{\s}
\]

\[
	\begin{aligned}
	  \Fib[n + 1] & = \frac{(\varphi^n - \psi^n)}{\s} + \frac{\varphi^{n - 1} - \psi^{n - 1}}{\s} \\
	  &= \frac{\varphi^n - \psi^n + \varphi^{n - 1} - \psi^{n - 1}}{\s} \\
	  &= \frac{\varphi^n + \varphi^{n - 1} - \psi^n - \psi^{n - 1}}{\s} \\
	  &= \frac{\varphi^{n-1} (\varphi + 1) - (\psi^{n -1} (\psi + 1))}{\s}
   \end{aligned}
\]

We would like $\varphi + 1 = \varphi^2$, is it so?
\[
	\begin{aligned}
	  \varphi + 1 &= \frac{1 + \s}{2} + 1\\
	  &= \frac{2 (1 + \s)}{4} + \frac{4}{4}\\
	  &= \frac{2 + 2\s + 4}{4} \\
	  &= \frac{1 + 1 + 2\s + 4}{4} \\
	  &= \frac{1 + 2\s + 5}{4} \\
	  &= \frac{(1 + \s)^2}{2^2} \\
	  &= \left(\frac{1 + \s}{2} \right)^2 \\
	  &= \varphi^2
   \end{aligned}
\]
By analogy we see that $\psi + 1$ will be equal to $\psi^2$.

\[
\begin{aligned}
	  \Fib[n + 1] &= \frac{\varphi^{n-1} (\varphi + 1) - (\psi^{n -1} (\psi + 1))}{\s} \\
		&= \frac{\varphi^{n-1} \varphi^2 - \psi^{n -1} \psi^2}{\s} \\
		&= \frac{\varphi^{n+1} - \psi^{n + 1}}{\s} 
\end{aligned}
\]

QED

\section{Proove that $\Fib[n]$ is the closest integer to $\frac{\varphi^n}{\s}$}

We know that:
\[
	\Fib[n] = \frac{\varphi^n + \psi^n}{\s}
\]

We need to prove that the distance between $\Fib[n]$ and $\frac{\varphi^n}{\s}$
is less or equal to $\frac{1}{2}$ (if it's more then there exists an integer that
s closer).

\[
	\begin{aligned}
		\left | \Fib[n] - \frac{\varphi^n}{\s} \right| & \le \frac{1}{2} \\
		\left | \frac{\varphi^n + \psi^n}{\s}  - \frac{\varphi^n}{\s} \right| & \le \frac{1}{2} \\
		\left | \frac{ \psi^n}{\s} \right| & \le \frac{1}{2} \\
		\left | \psi^n \right| & \le \frac{\s}{2} \\
	\end{aligned}
\]

\[
	\frac{\sqrt{5}}{2} \approx 1.12 > 1
\]
\[
   \left | \frac{1 - \s}{2} \right| \approx 0.62 < 1
\]

From exponentation we know what if $x < 1$ then $x^n < 1 \; \forall n > 0$

\[
   \left | \phi^n \right| < \frac{\sqrt{5}}{2}
\]

Is true.

% (

\end{document}
