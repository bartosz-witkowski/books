\documentclass{article}

\usepackage{amsmath}

\newcommand \w[0]{\text{width}}

\begin{document}

\[
   \w([a, b]) = \frac{b - a}{2}.
\]

Sum:

\[
\begin{aligned}
  \, [a, b] + [c, d] & = [a + c, b + d] \\
   \w([a + c, b + d]) &= \frac{b + d - (a + c)}{2} \\
      & = \frac{b - a}{2} + \frac{d - c}{2} \\
      & = \w([a, b]) + \w([c, d])
\end{aligned}
\]

Difference:

\[
\begin{aligned}
   \, [a, b] - [c, d] & = [a - d, b - c] \\
   \w([a - d, b - c]) &= \frac{b - c - (a - d)}{2} \\
      & = \frac{b - a -  c + d}{2} \\
      & = \frac{b - a}{2} + \frac{d - c}{2} \\
      & = \w([a,b]) + \w([c, d])
\end{aligned}
\]

Multiplication:

\[
   \begin{aligned}
      \w([1, 2]) & = \frac{1}{2} \\
      \w([2, 4]) & = 1 \\
      \\
      \w([1, 2] \cdot [2, 4]) & = \w( 
         [\min(1 \cdot 2, 1 \cdot 4, 2 \cdot 2, 2 \cdot 4,
          \max(1 \cdot 2, 1 \cdot 4, 2 \cdot 2, 2 \cdot 4
         ]
      ) \\
      & = \w([2, 8]) \\
      & = 3
   \end{aligned}
\]

We can find another interval with the width $1$ for example

$\w([5, 7]) = 1$ and now 

\[
   \w([1, 2] \cdot [5, 7]) = \w([5, 14]) = 4.5
\]

The same principle applies to division.
\end{document}
