\documentclass{article}

\usepackage{amsmath}
\usepackage{amsfonts}

\newcommand{\n}{\texttt{natural\_number}}

\begin{document}

A not so rigorous proof by induction:

For $n, m = 0$:

\[
   \begin{aligned}
      \le(0, 0) & \longrightarrow \n(0) \\
         & \longrightarrow 0 
   \end{aligned}
\]

Proof tree depth = 2.

For $m > 0$ and for some $n_1 \le m$:

\[
   \begin{aligned}
      \le(s^{n_1}(0), s^{m}(0)) & \longrightarrow \le(s^{n_1-1}(0), s^{m-1}(0)) \\
         & \longrightarrow  \le(s^{n_1-2}(0), \le s^{m-2}(0)) \\
         & \ldots \\
         & \longrightarrow \le(s(0), s^{m-n_1+1}(0)) \\
         & \longrightarrow \le(0,  s^{m-n_1}(0)) \\
         & \longrightarrow \n(s^{m-n_1}) \\
         & \ldots \\
         & \longrightarrow \n(s(0)) \\
         & \longrightarrow \n(0) \\
   \end{aligned}
\]

$k = m - 1$

\[
   \begin{aligned}
      \le(s^{n_2}(0), s^{k}(0)) & \longrightarrow \le(s^{n_2-1}(0), s^{k-1}(0)) \\
         & \longrightarrow  \le(s^{n_2-2}(0), \le s^{k-2}(0)) \\
         & \ldots \\
         & \longrightarrow \le(s(0), s^{k-n_2+1}(0)) \\
         & \longrightarrow \le(0,  s^{k-n_2}(0)) \\
         & \longrightarrow \n(s^{k-n_2}) \\
         & \ldots \\
         & \longrightarrow \n_2(s(0)) \\
         & \longrightarrow \n_2(0) \\
   \end{aligned}
\]

Which is just:

\[
   \begin{aligned}
      \le(s^{n_2}(0), s^{m-1}(0)) & \longrightarrow \le(s^{n_2-1}(0), s^{m-2}(0)) \\
         & \longrightarrow  \le(s^{n_2-2}(0), \le s^{m-3}(0)) \\
         & \ldots \\
         & \longrightarrow \le(s(0), s^{m-n_2+1}(0)) \\
         & \longrightarrow \le(0,  s^{m-n_2}(0)) \\
         & \longrightarrow \n(s^{m-n_2}) \\
         & \ldots \\
         & \longrightarrow \n_2(s(0)) \\
         & \longrightarrow \n_2(0) \\
   \end{aligned}
\]

The trees for $k$ and $m$ are the same but one step so by the induction
hypothesis the 

\end{document}
